\documentclass{article}
\usepackage{arxiv}
\usepackage[utf8]{inputenc}
\usepackage[english, russian]{babel}
\usepackage[T1]{fontenc}
\usepackage{url}
\usepackage{booktabs}
\usepackage{amsfonts}
\usepackage{nicefrac}
\usepackage{microtype}
\usepackage{lipsum}
\usepackage{graphicx}
\usepackage{natbib}
\usepackage{doi}




\author{ Sergei Anikin \\
% \thanks{Use footnote for providing further
% 		information about author (webpage, alternative
% 		address)---\emph{not} for acknowledging funding agencies.} \\
	Chair of Data Analysis\\
	MIPT\\
    Moscow, Russia\\
	% Pittsburgh, PA 15213 \\
	\texttt{anikin.sd@phystech.edu} \\
	%% examples of more authors
	\And
	Alexandr Bulkin \\
	MSU \\
    Faculty of Mechanics and Mathematics \\
	Moscow, Russia\\
    \texttt{a.bulkin@iccda.io} \\
	% Santa Narimana, Levand \\
	% \texttt{stariate@ee.mount-sheikh.edu} \\
	%% \AND
	%% Coauthor \\
	%% Affiliation \\
	%% Address \\
	%% \texttt{email} \\
	%% \And
	%% Coauthor \\
	%% Affiliation \\
	%% Address \\
	%% \texttt{email} \\
	%% \And
	%% Coauthor \\
	%% Affiliation \\
	%% Address \\
	%% \texttt{email} \\
}
\date{}

\renewcommand{\shorttitle}{\textit{arXiv} Template}

%%% Add PDF metadata to help others organize their library
%%% Once the PDF is generated, you can check the metadata with
%%% $ pdfinfo template.pdf
\hypersetup{
pdftitle={A template for the arxiv style},
pdfsubject={q-bio.NC, q-bio.QM},
pdfauthor={David S.~Hippocampus, Elias D.~Striatum},
pdfkeywords={First keyword, Second keyword, More},
}

\begin{document}
\maketitle

\begin{abstract}
	This paper addresses the well-known Max Cut problem, which has various applications both in machine learning and theoretical physics. The Max Cut problem is computationally intractable over general graphs. This paper presents a novel empirical approach aimed at enhancing the quality of Max-Cut approximations within polynomial time bounds. While the problem is tractable for graphs with small tree-width, its solution over general graphs often relies on Semi-Definite Programming or Lovász-Schrijver relaxations. We achieve the described improvement of approximation quality by combining relaxation approaches, the tree-width ideas and various heuristics described in the paper.




\end{abstract}


\keywords{SDP \and Treewidth \and Max Cut \and Lovász-Schrijver relaxations}

% \section{Introduction}
% \lipsum[2]
% \lipsum[3]

% \section{Headings: first level}
% \label{sec:headings}

% \lipsum[4] See Section \ref{sec:headings}.

% \subsection{Headings: second level}
% \lipsum[5]
% \begin{equation}
% 	\xi _{ij}(t)=P(x_{t}=i,x_{t+1}=j|y,v,w;\theta)= {\frac {\alpha _{i}(t)a^{w_t}_{ij}\beta _{j}(t+1)b^{v_{t+1}}_{j}(y_{t+1})}{\sum _{i=1}^{N} \sum _{j=1}^{N} \alpha _{i}(t)a^{w_t}_{ij}\beta _{j}(t+1)b^{v_{t+1}}_{j}(y_{t+1})}}
% \end{equation}

% \subsubsection{Headings: third level}
% \lipsum[6]

% \paragraph{Paragraph}
% \lipsum[7]



% \section{Examples of citations, figures, tables, references}
% \label{sec:others}

% \subsection{Citations}
% Citations use \verb+natbib+. The documentation may be found at
% \begin{center}
% 	\url{http://mirrors.ctan.org/macros/latex/contrib/natbib/natnotes.pdf}
% \end{center}

% Here is an example usage of the two main commands (\verb+citet+ and \verb+citep+): Some people thought a thing \citep{kour2014real, hadash2018estimate} but other people thought something else \citep{kour2014fast}. Many people have speculated that if we knew exactly why \citet{kour2014fast} thought this\dots

% \subsection{Figures}
% \lipsum[10]
% See Figure \ref{fig:fig1}. Here is how you add footnotes. \footnote{Sample of the first footnote.}
% \lipsum[11]

% \begin{figure}
% 	\centering
% 	\includegraphics[width=0.5\textwidth]{../figures/log_reg_cs_exp.eps}
% 	\caption{Sample figure caption.}
% 	\label{fig:fig1}
% \end{figure}

% \subsection{Tables}
% See awesome Table~\ref{tab:table}.

% The documentation for \verb+booktabs+ (`Publication quality tables in LaTeX') is available from:
% \begin{center}
% 	\url{https://www.ctan.org/pkg/booktabs}
% \end{center}


% \begin{table}
% 	\caption{Sample table title}
% 	\centering
% 	\begin{tabular}{lll}
% 		\toprule
% 		\multicolumn{2}{c}{Part}                   \\
% 		\cmidrule(r){1-2}
% 		Name     & Description     & Size ($\mu$m) \\
% 		\midrule
% 		Dendrite & Input terminal  & $\sim$100     \\
% 		Axon     & Output terminal & $\sim$10      \\
% 		Soma     & Cell body       & up to $10^6$  \\
% 		\bottomrule
% 	\end{tabular}
% 	\label{tab:table}
% \end{table}

% \subsection{Lists}
% \begin{itemize}
% 	\item Lorem ipsum dolor sit amet
% 	\item consectetur adipiscing elit.
% 	\item Aliquam dignissim blandit est, in dictum tortor gravida eget. In ac rutrum magna.
% \end{itemize}


\bibliographystyle{unsrtnat}
\bibliography{references}

[1] Goemans-Williamson MAXCUT Approximation Algorithm by Jin-Yi Cai, Christopher Hudzik, Sarah Knoop, 2003:
https://pages.cs.wisc.edu/~jyc/02-810notes/lecture20.pdf \\

[2] The Lovasz-Schrijver relaxation by Madhur Tulsiani, 2010: 
https://home.ttic.edu/~madhurt/Papers/ls.pdf \\

% [3] Introduction to the SDP by Robert M. Freund, 2004: https://ocw.mit.edu/courses/15-084j-nonlinear-programming-spring-2004/a632b565602fd2eb3be574c537eea095_lec23_semidef_opt.pdf\\

[4] Treewidth: 
https://www.cs.cmu.edu/~odonnell/toolkit13/lecture17.pdf\\

[5] MAX CUT approximation algorithm and UGC-hardness, Lecture by Irit Dinur and Amey Bhangale:
https://www.wisdom.weizmann.ac.il/~dinuri/courses/19-inapprox/lec6.pdf \\

[6] Semidefinite Programming versus Burer-Monteiro Factorization for Matrix
Sensing by Baturalp Yalcin, Ziye Ma, Javad Lavaei, Somayeh Sojoudi, 2022
https://arxiv.org/abs/2208.07469v1\\

[7] 0.878-approximation for the Max-Cut problem, Lecture by Divya Padmanabhan, 2022: https://www.iitgoa.ac.in/~sreejithav/misc/maxcut.pdf\\

[8] Rank optimality for the Burer-Monteiro factorization by Irène Waldspurger, Alden Waters, 2019 
https://arxiv.org/abs/1812.03046\\

[9] Semidefinite relaxation and nonconvex quadratic optimization by Yury Nesterov, 1997
https://www.tandfonline.com/doi/abs/10.1080/10556789808805690\\

[10] Datasets: Texas Data Repository:
https://dataverse.tdl.org/dataset.xhtml?persistentId=doi:10.18738/T8/VLTIVC\\  

[11] Datasets: Biq Mac Library:
https://biqmac.aau.at/biqmaclib.html\\ 

[12] Datasets: MaxCut and BQP Instance Library:
http://bqp.cs.uni-bonn.de/library/html/index.html\\ 

[13] Datasets: MaxCut Instances:
https://grafo.etsii.urjc.es/optsicom/maxcut.html#best-known-values\\





\end{document}
